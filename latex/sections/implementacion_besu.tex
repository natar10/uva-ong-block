\section{Implementación en Besu}


El contrato inteligente fue desarrollado en Solidity\cite{SolidityDocs} y desplegado en una red Hyperledger Besu\cite{HyperledgerBesu} utilizando Remix IDE\cite{RemixIDE}. Esta sección describe la arquitectura del contrato, sus funciones principales y las consideraciones técnicas encontradas durante el despliegue.

\subsection{Arquitectura del contrato}

El contrato \texttt{ONGDonaciones.sol} implementa un diseño modular, donde  cada parte del contrato ( donación, compra del material, vatciones, ...) viene definida en un ''subcontrato'' diferente para una mejor organización.
De esta manera, en el contrato \texttt{ContratoONG.sol} únicamente se compila el resto de contratos, vinculados a éste por herencias

con las siguientes estructuras de datos:

\begin{itemize}
    \item \textbf{Estructuras (Structs):}
    \begin{itemize}
        \item \texttt{Donante}: almacena dirección, nombre, tipo, total donado y tokens de gobernanza.
        \item \texttt{Proyecto}: contiene ID, descripción, responsable, cantidades recaudada y validada, estado y votos.
        \item \texttt{Donacion}: registra ID, donante, proyecto, cantidad y timestamp.
        \item \texttt{Compra}:
        \item \texttt{Proveedor}:
    \end{itemize}

    \item \textbf{Mapeos (Mappings):}
    \begin{itemize}
        \item \texttt{mapping(address => Donante) donantes}
        \item \texttt{mapping(string => Proyecto) proyectos}
        \item \texttt{mapping(string => Donacion) donaciones}
    \end{itemize}

    \item \textbf{Arrays para iteración:}
    \begin{itemize}
        \item \texttt{address[] listaDonantes}
        \item \texttt{string[] listaProyectos}
        \item \texttt{string[] listaDonaciones}
    \end{itemize}
\end{itemize}

Esta combinación de mappings para acceso directo y arrays para iteración permite tanto búsquedas eficientes como la capacidad de listar todos los registros desde el frontend.

\subsection{Funciones principales}

El contrato implementa las siguientes operaciones:

\subsubsection{Funciones de escritura}

\begin{itemize}
    \item \texttt{registrarDonante(string nombre, TipoDonante tipo)}: Permite a un usuario registrarse como donante. Verifica que no esté previamente registrado.

    \item \texttt{crearProyecto(string id, string descripcion, address responsable)}: Función restringida al owner para crear nuevos proyectos. Valida que el ID no exista previamente.

    \item \texttt{donar(string proyectoId) payable}: Función principal que procesa donaciones. Recibe ETH, registra la donación, actualiza totales y asigna tokens de gobernanza (1 token por ETH). Si el donante no está registrado, lo crea automáticamente como individuo.

    \item \texttt{votarProyecto(string proyectoId, uint256 cantidadVotos)}: Permite a los donantes gastar sus tokens para votar por proyectos. Verifica que el donante tenga tokens suficientes y que el proyecto esté activo.

    \item \texttt{validarFondosProyecto(string proyectoId, uint256 cantidad)}: Función administrativa para marcar fondos como validados/gastados.

    
\end{itemize}

\subsubsection{Funciones de lectura}

El contrato incluye funciones de consulta que no modifican el estado:

\begin{itemize}
    \item \texttt{obtenerDonante(address)}, \texttt{obtenerProyecto(string)}, \texttt{obtenerDonacion(string)}
    \item \texttt{obtenerTotalDonantes()}, \texttt{obtenerTotalProyectos()}, \texttt{obtenerTotalDonaciones()}
    \item \texttt{obtenerBalance()}: retorna el balance de ETH del contrato
\end{itemize}

\subsection{Control de acceso}

Se implementa un sistema de permisos básico mediante el modificador \texttt{soloOwner}, que restringe funciones críticas al propietario del contrato:

\begin{lstlisting}[language=Solidity]
modifier soloOwner() {
    require(msg.sender == owner,
            "Solo el owner puede ejecutar esto");
    _;
}
\end{lstlisting}

El owner se establece en el constructor como la dirección que despliega el contrato.

\subsection{Eventos}

Para facilitar la integración con el frontend y permitir el seguimiento de operaciones, se definieron los siguientes eventos:

\begin{itemize}
    \item \texttt{DonacionRealizada(address indexed donante, string proyectoId, uint256 cantidad)}
    \item \texttt{ProyectoCreado(string id, string descripcion)}
    \item \texttt{DonanteRegistrado(address indexed direccion, string nombre)}
    \item \texttt{VotacionRealizada(address indexed donante, string proyectoId, uint256 cantidad\_votos)}
    \item \texttt{CompraRealizada(address indexed donante, string compradorId, uint256 valor\_compra))}

\end{itemize}

\subsection{Proceso de despliegue}

El contrato fue desplegado utilizando Remix IDE conectado a la red BLOCK LAB de Besu. El proceso incluyó:

\begin{enumerate}
    \item Compilación del contrato con Solidity 0.8.0.
    \item Conexión de MetaMask\cite{MetaMask} a la red BLOCK LAB.
    \item Configuración de permisos de dominio en MetaMask.
    \item Despliegue mediante Remix utilizando el injected provider.
    \item Obtención de la dirección del contrato desplegado.
\end{enumerate}

\subsection{Consideraciones técnicas importantes}

Durante la implementación y despliegue se identificaron varios aspectos críticos:

\begin{itemize}
    \item \textbf{Permisos de red en MetaMask:} Es fundamental configurar correctamente los permisos para el dominio específico en MetaMask, asegurando que la extensión utilice la red BLOCK LAB. Sin esta configuración, suelen aparecer problemas de conexión.

    \item \textbf{Inmutabilidad de contratos:} Los contratos en blockchain no son editables una vez desplegados. Cualquier modificación requiere desplegar un nuevo contrato con una nueva dirección, lo que implica actualizar la configuración en el frontend.

    \item \textbf{Coordinación frontend-contrato:} La dirección del contrato debe actualizarse manualmente en el archivo de configuración del frontend cada vez que se despliega una nueva versión, usando para ello el archivo ABI generado al desplegar el contrato.

    \item \textbf{Gas y costos:} Aunque Besu no requiere ETH real, se debe tener en cuenta el concepto de gas para futuras implementaciones en redes públicas.
\end{itemize}
