

\subsection{Repositorio y página web}
El código fuente de la aplicación está disponible en el siguiente repositorio de GitHub: 

\href{https:\/\/github.com\/natar10\/uva-ong-block}{https:\/\/github.com\/natar10\/uva-ong-block}

Además, se puede visitar e interactuar con la aplicación a través del siguiente enlace:

\href{https:\/\/uva-ong-block.vercel.app\/}{https:\/\/uva-ong-block.vercel.app\/}


\subsection{\textit{Walkthrough} de la aplicación}\label{app:walkthrough}

Primero, el usuario \textit{Donante} se registra en la plataforma, proporcionando un nombre. Esto hará saltar la extensión de Metamask, donde el usuario deberá seleccionar la cuenta con la que se registra a la red blockchain.

\begin{figure}[H]
    \centering
    \includegraphics[width=0.6\textwidth]{./assets/screenshots/RegistroDonante.png}
\end{figure}

Una vez registrado, el usuario puede empezar a visualizar y donar a los diferentes proyectos de la ONG:

\begin{figure}[H]
    \centering
    \includegraphics[width=0.7\textwidth]{./assets/screenshots/donacion.png}

    \includegraphics[width=0.7\textwidth]{./assets/screenshots/historico.png}
\end{figure}


Gracias a las donaciones, el donante recibirá votos, que podrá usar para votar para que se aprueben nuevos proyectos, cuando se supere el margen establecido.

\begin{figure}[H]
    \centering
    \includegraphics[width=0.7\textwidth]{./assets/screenshots/votacion.png}
    \includegraphics[width=0.7\textwidth]{./assets/screenshots/registro_voto.png}

    \includegraphics[width=0.7\textwidth]{./assets/screenshots/votacion.png}
\end{figure}


\subsection{Walkthrough \textit{Remix}}\label{app:despliegue_remix}

El deploy de la aplicación se realizó a través de Remix IDE, compilándolo con el compilador usando el evm versión \textit{London}.

\begin{figure}[H]
    \centering
    \includegraphics[width=0.7\textwidth]{./assets/screenshots_remix/DeployRemix.png}
\end{figure}


Para ello, hay que hacer el deploy del contrato en un orden concreto, ya que primero hay que inicializar el contrato del token ERC20 sobre el que se moverán los votos, para a continuación desplegar el contrato principal de la ONG pasándole la dirección del contrato por el que se gestionarán los tokens de Gobernanza.



\begin{figure}[H]
    \centering
    \includegraphics[width=0.7\textwidth]{./assets/screenshots_remix/ToeknGobernanza.png}


    \includegraphics[width=0.7\textwidth]{./assets/screenshots_remix/ContratoONG.png}
\end{figure}


\subsection{Token de Gobernanza}\label{app:token_gobernanza}

\begin{figure}[H]
    \centering
    \includegraphics[width=0.7\textwidth]{./assets/screenshots/tokenGobernanza.png}
\end{figure}
