\section{Simplificación en base al diseño inicial}

Durante la fase conceptual se diseñó un sistema ambicioso que incluía múltiples actores, mecanismos complejos de distribución de fondos, gestión de proveedores y un sistema de votaciones con ciclos anuales. Sin embargo, al enfrentar la implementación práctica se tomaron decisiones de simplificación para garantizar un sistema funcional y comprensible.

\subsection{Elementos eliminados del diseño original}

El modelo conceptual contemplaba diversos componentes que fueron descartados en la implementación:

\begin{itemize}
    \item \textbf{Gestión de proveedores:} El diseño inicial incluía un registro de proveedores autorizados (constructoras, distribuidores de material educativo, editoriales) con historial de compras. Este mecanismo fue modificado para simplificar el flujo de fondos. En la versión a actual, todos los proveedores venden todo tipo de material, y el precio es estático, fijado en el contrato. 

    \item \textbf{Administración ONG con billetera separada:} Se planeaba una cuenta administrativa que recibiría automáticamente el 20\% de cada donación para gastos logísticos. Esta funcionalidad no se implementó.

    \item \textbf{Distribución porcentual automática:} El sistema original contemplaba que cada donación se distribuyera según porcentajes predefinidos (20\% a administración, 30\% equitativo entre proyectos sin meta, 50\% según votaciones). Este mecanismo complejo fue descartado.

    \item \textbf{Ciclos de votación y quema de tokens:} Se diseñó un sistema de votaciones anuales donde los tokens no utilizados se quemaban al final del ciclo. La implementación actual permite votaciones continuas sin restricciones temporales.

    \item \textbf{Objetivo de los tokens de votación:} En la misma línea, los tokens de gobernanza distribuidas ahora tienen una utilidad diferente. En vez de utilizarse para asignar los porcentajes de la donación que se lleva cada proyecto, ahora se utilizan para votar a favor/en contra de proyectos, que necesitan tener un balance de votos positivos determinado para estar activos (cuando un proyecto deja de estar activo, la cantidad donada sin validar se devuelve).

    \item \textbf{Metas de donación y redistribución:} El diseño conceptual incluía metas para cada proyecto y redistribución automática de fondos cuando un proyecto era cancelado. Estas reglas no fueron implementadas.
    
    \item \textbf{Distinción entre tipos de donante para tokens:} Aunque el contrato mantiene el enum \texttt{TipoDonante}, no se implementó la regla diferencial de asignación de tokens (raíz cuadrada para empresas vs. proporcional para individuos).
\end{itemize}

\subsection{Sistema implementado}

El contrato final \texttt{ONGDonaciones.sol} se centra en las funcionalidades esenciales:

\begin{itemize}
    \item \textbf{Estructura de datos simplificada:} Se mantienen tres estructuras principales: \texttt{Donante}, \texttt{Proyecto} y \texttt{Donacion}, eliminando la complejidad de proveedores y administración.

    \item \textbf{Flujo directo de donaciones:} Los fondos van directamente al proyecto especificado sin distribución porcentual automática.

    \item \textbf{Tokens de gobernanza básicos:} Se asigna 1 token por cada ETH donado, sin distinción por tipo de donante.

    \item \textbf{Votación simplificada:} Los donantes pueden votar en cualquier momento, gastando sus tokens para incrementar o disminuir el contador de votos del proyecto.

    \item \textbf{Control de acceso básico:} Solo el owner puede crear proyectos, validar fondos y cambiar estados, implementado mediante el modificador \texttt{soloOwner}.
\end{itemize}

\subsection{Justificación de las simplificaciones}

Estas decisiones responden a criterios de viabilidad técnica y pedagógica:

\begin{enumerate}
    \item \textbf{Reducción de complejidad:} Un contrato más simple es más fácil de auditar, probar y mantener.
    \item \textbf{Enfoque en funcionalidades core:} Se priorizaron las capacidades esenciales de trazabilidad y gobernanza.
    \item \textbf{Facilita la comprensión:} Un sistema menos complejo permite entender mejor los conceptos fundamentales de blockchain.
    \item \textbf{Menor superficie de ataque:} Menos código significa menos vectores potenciales de vulnerabilidades.
\end{enumerate}

A pesar de estas simplificaciones, el sistema implementado cumple con los objetivos principales: proporcionar transparencia en las donaciones y permitir la participación democrática mediante votaciones.
